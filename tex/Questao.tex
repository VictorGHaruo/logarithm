\section{Questão Proposta}

Seja $f:\mathbb{R}_{>0} \to \mathbb{R}$ função contínua tal que $f(xy) = f(x) + f(y)$, o objetivo desta questão é mostrar que $f(x) = f(e)\text{ln}(x)$.
\begin{enumerate}[(a)]
    \item Mostre que $f(e^n) = n\cdot f(e) \quad \forall n \in \mathbb{N}$
    \item Mostre que $f(e^q) = q\cdot f(e) \quad \forall q \in \mathbb{Q}_{>0}$
    \item Conclua por continuidade que vale $f(e^r) = r\cdot f(e) \, \forall r \in \mathbb{R}_{>0}$ e portanto deve-se ter $f(x) = f(e) \text{ln}(x)$.
\end{enumerate}

\noindent Dica: Para o item (c) você deve utilizar dois resultados importantes vistos em um curso de análise real:
\begin{enumerate}
    \item $f$ é contínua em $a$ se para toda sequência $x_n$ convergente a $a$, tem-se $\lim f(x_n) = f(a)$
    \item Os racionais são densos na reta, isto é, para todo número real $r$, existe uma sequência $q_n$ de racionais convergente a $r$
\end{enumerate}
