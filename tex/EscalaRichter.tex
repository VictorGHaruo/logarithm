\subsection{Escala Richter: Da Amplitude à Escala Destrutiva}
Durante os noticiários sobre desastres naturais, frequentemente a intensidade dos terremotos é quantificada por um número na Escala de Magnitude Richter com o intuito de evidenciar a magnitude da perda material e humanitária encadeada pela energia liberada do tremor.

Mas afinal, como surgiu e como interpreto a escala? Desenvolvida em meados da década de 1930 pelos pesquisadores Charles Richter e Beno Gutenberg, da Caltech, a Escala Richter originou-se da necessidade de condensar um conjunto denso de dados de um sismógrafo em um único número, de forma concisa e intuitiva. Para isso, propuseram a seguinte fórmula:

\begin{equation}
    M = \log_{10}\left(\frac{A}{A_0}\right)
\end{equation}

onde $A$ é a amplitude máxima da onda sísmica registrada e $A_0$ é uma amplitude de referência. É importante notar que $A_0$ não é uma constante; seu valor depende da distância entre o sismógrafo e o epicentro do terremoto.

Em virtude da natureza logarítmica da fórmula, um acréscimo de apenas uma unidade na escala não representa um aumento linear, mas sim um aumento exponencial na intensidade do tremor. Aplicando a definição de logaritmo, podemos isolar a amplitude $A$:

\begin{equation}
    M = \log_{10}\left(\frac{A}{A_0}\right) \iff A = 10^{M} \cdot A_0
\end{equation}

Para compreender melhor, podemos comparar a amplitude de um terremoto de magnitude $M$ (denotaremos a amplitude do tremor com magnitude $M$ por meio de $A_M$) com a de um terremoto de magnitude $M+1$:

\begin{equation*}
\begin{cases}
    A_M = 10^{M} \cdot A_0 \\
    A_{M+1} = 10^{M+1} \cdot A_0
\end{cases}
\quad\Longrightarrow\quad A_{M+1} = 10 \cdot A_M
\end{equation*}

Daí, vemos que o aumento de apenas um ponto na Escala Richter corresponde a um tremor com amplitude 10 vezes maior.

Ademais, perceba que tal resultado é referente ao tremor e não necessariamente sobre sua capacidade destrutiva. Dessa maneira, a fim de ilustrar o potencial destrutivo do acréscimo de uma unidade na Escala Richter introduziremos o conceito de energia:

\begin{equation*}
    E_M \propto A^{3/2} \iff E_M = k \cdot A^{3/2}
\end{equation*}

Denotando a energia liberada por um terremoto de escala $M+1$ por $E_{M+1}$, temos que:

\begin{align*}
    E_{M+1} &\propto (A_{M+1})^{3/2} \\
            &= (10 \cdot A_M)^{3/2} \\
            &= 10^{3/2} \cdot (A_M)^{3/2} \\
            &\propto 10^{3/2} \cdot E_M
\end{align*}

Perceba que, $10^{3/2} \approx 31,6$, isto é, a capacidade destrutiva do tremor é 31,6 vezes maior ao aumentar uma unidade na escala, e o impacto sobre a realidade pode ser percebido através da seguinte tabela:

\begin{table}[H]
\centering
\label{tab:richter_simples}
\begin{tabular}{|c|l|}
    \hline
\textbf{Magnitude} & \textbf{Descrição dos Efeitos} \\
\hline
1--3 & Nada / Imperceptível \\
3--4 & Perceptível, sem danos \\
4--5 & Danos domésticos \\
5--6 & Danos em prédios sem estrutura adequada \\
6--7 & Danos em prédios normais \\
7--8 & Danos sérios \\
8--9 & Destruição de cidades \\
9--10 & Muda a topografia \\
$>10$ & Desconhecido / Catástrofe global \\
\hline
\end{tabular}
\caption{Efeitos por Magnitude na Escala Richter}
\end{table}



Fica claro, portanto, que o conceito introduzido por John Napier em 1614 transcende a matemática abstrata. O real conhecimento dos logaritmos permite uma análise crítica da realidade. Diferentemente do que o senso comum poderia sugerir, um terremoto de magnitude 10 não causa o dobro da destruição de um de magnitude 5. Na verdade, seu poder destrutivo é cerca de 31 milhões de vezes maior. Ter essa percepção nos ajuda a compreender melhor a magnitude dos fenômenos que nos rodeiam e analisá-los com senso crítico.
