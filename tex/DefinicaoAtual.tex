\section{Definição Atual}

Apesar da enorme importância dos logaritmos para cálculos numéricos, não é dessa maneira que somos introduzidos aos logaritmos na escola. Na verdade, a definição de logaritmo que aprendemos é intimamente ligada a uma exponencial. A definição é a que segue

\textbf{Definição.} Sendo $a$ e $b$ números reais e positivos, com $a \neq 1$, chama-se \textit{logaritmo} de $b$ na base $a$ o expoente que se deve dar à base $a$ de modo que a potência obtida seja igual a $b$.
Em símbolos: se $a, b \in \mathbb{R}$, $0 < a \neq 1$ e $b > 0$, então:

\[
\log_{a} b = x \iff a^x = b
\]

A partir desta definição, é tranquilo chegarmos em algumas propriedades interessantes, não iremos prová-las pois não está no escopo do texto, porém, é um excelente exercício para o leitor.

\vspace{8pt}
\noindent\textbf{\underline{Propriedades dos logaritmos:}}

\begin{multicols}{2}
\begin{enumerate}
    \item ${a^{\log_a x} = x}$  
    \item ${\log_a a^k = k}$  
    \item ${\log_a 1 = 0}$ 
    \item ${\log_a a = 1}$ 
    \item ${\log_a x^k = k \log_a x}$ 
    \item ${\log_a(x\cdot y) = \log_a x + \log_a y}$ 
    \item ${\log_a\!\left(\dfrac{x}{y}\right) = \log_a x - \log_a y}$
    \item ${\log_a x = \dfrac{\log_b x}{\log_b a}}$ 
\end{enumerate}
\end{multicols}


